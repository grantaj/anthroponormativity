\documentclass[12pt]{article}
\usepackage{graphicx}
\usepackage{setspace}
\usepackage[utf8]{inputenc}
\usepackage{epigraph}
\usepackage{titlesec}

\titleformat{\section}
  [block]
  {\filcenter\normalfont\scshape}
  {}
  {0pt}
  {}

\titleformat{\subsection}
  [block]
  {\raggedright\normalfont\scshape\normalsize}
  {}
  {0pt}
  {}

\titleformat{\subsubsection}
  [block]
  {\raggedright\normalfont\itshape}
  {}
  {0pt}
  {}

\makeatletter
\renewcommand{\maketitle}{%
  \begin{center}
    {\LARGE\scshape\@title \par}
    \vskip 1em
    {\large\scshape\@author \par}
    \vskip 1em
    {\normalsize\@date \par}
  \end{center}
}
\makeatother

\setlength{\epigraphwidth}{\textwidth}
\setlength{\epigraphrule}{0pt}
\renewcommand{\textflush}{flushleft}
\renewcommand{\epigraphsize}{\itshape\normalsize}
\renewcommand{\epigraphflush}{flushleft}
\newcommand{\epigraphsource}[1]{--- {\small{\textit{#1}}}}

\usepackage[
  backend=biber,
  notes,
  dashed=false,
  doi=false,
  url=true,
  isbn=false,
  giveninits=true
]{biblatex-chicago}
\addbibresource{main.bib}

\interfootnotelinepenalty=10000

\title{Anthroponormativity}
\author{Alex Grant}
\date{\today}

\begin{document}
\onehalfspacing
\maketitle

\begin{abstract}
Modernity has been shaped by a progressive sequence of decentrings
that dislodge humanity from a position of assumed privilege. The rise of
machine cognition intensifies this trajectory, revealing intelligence as
an emergent and distributed process rather than a uniquely human
essence. Anthroponormativity describes the prevailing assumption of the centrality of human intelligence in our definitions of mind. This assumption induces mind-blindness, an invisible bias against
diverse ``unexpected minds''. Treating decentring as a generative
shift rather than a loss, I advocate a stance of ``extreme empathy'' that
resists mystification while widening moral and cultural attention to
other minds. I position art-making as a living cultural
process -- a social activity performed by artists and their
communities, rather than the product of isolated genius. On this view,
the question is no longer whether machines threaten creativity, but what
forms of relationship and cultural life open when intelligence is
understood as plural.
\end{abstract}

\newpage 

\section{Three Great De-centrings}
Decentring is a recurring historical pattern: a crisis of meaning
followed by an expansion of understanding.\autocite{kuhn1970} I consider
a sequence of decentrings from Copernicus to Darwin to Turing
that progressively reframe the position of humanity in our view of the
world.

\subsection{Copernicus}

The pre-Copernican worldview was a geocentric
Aristotelian--Ptolemaic cosmos, with nested spheres carrying the Moon,
planets, and stars, a perfect and unchanging heaven, and a corruptible
terrestrial realm below. This cosmological exceptionalism was integrated into Christian
social order and theology, lending humanity a privileged centre.
\autocite{dreyer1906,grant2001} In that regime, the advance of knowledge
was expected to fit the doxa of divine creation and the Aristotelian
hierarchy rather than revise it. Copernicus began the reversal: a slow
shift toward knowledge proceeding from empirical and mathematical
coherence, even when it violated inherited authority.

Copernicus revealed that this centre was a mistake. Heliocentrism
reorganised the frame of reference itself, demonstrating that coherence
emerges not by adding complexity but by altering the frame.
\autocite{copernicus1543,kuhn1957} His 1543 proposal set the terms of
the shift. The consequences were not merely
astronomical but epistemic. Copernicus decoupled cosmology from theology
and displaced scripture methodologically: biblical cosmology could now
be treated as phenomenological rather than literal, while mathematical
coherence became the higher authority. The deeper shift was from
forcing philosophical and scientific thought to fit an unyielding
theological picture of the world to revising that picture in light of
observation and experiment. Scholastic authority was destabilised as
knowledge moved from inherited frameworks toward empirical and
mathematical testing.
\autocite{westman1975,brooke1991}

The disruption precedes overt theological conflict. Early reception was
muted by technical obscurity and cautious framing, yet the shift in
knowledge production was decisive: from inherited cosmological order to
constructed mathematical representation. Humanity's cosmic centrality
was implicitly undermined, and truth became increasingly specialised and
exclusionary.
\autocite{westman1975}
This conflict becomes explicit when heliocentrism is later defended by
Galileo as a claim about the physical motions of bodies rather than a
mere mathematical device, provoking ecclesiastical backlash; his 1633
trial and subsequent house arrest make the consequences concrete.
\autocite{brooke1991,westman1975}

On the scientific side, the lesson of frame is explicit. Epicycles\footnote{A mathematically elaborate system of deferents and epicycles that accumulated complexity to preserve geocentrism despite observational strain.}
were not merely a technical blemish; they were a symptom of a misaligned
frame of reference. Once the centre is reassigned, the mathematics
simplifies and a different kind of scientific practice emerges: one in
which coherence is purchased by changing the frame, not by adding ad hoc
complexity.
\autocite{dreyer1906} The same methodological shift echoes in later
physics. Einstein's insistence on the constancy of the speed of light
forces a new reference frame, yielding not only cleaner equations but an
entirely new ontology of space and time, and ultimately a new theory of
gravity in which curvature of spacetime replaces force.
\autocite{einstein1905,einstein1916,susskind2017}

For Westman, Copernicus destabilised religion and society not by
denying God, but by \emph{transforming how truth itself could be known}.
\autocite{westman1975}

\subsection{Darwin}

Darwin extends the decentring by dissolving biological exceptionalism.
His achievement is explanatory, not merely biological: evolution by
natural selection replaces divine design with an impersonal, law-like
process. Teleology is removed from nature, and human exceptionalism is
biologically undermined. Darwin's 1859 argument sets this decentring in
motion.\autocite{ruse1979,dennett1995}

This shift displaces creation narratives as sources of natural-historical
truth and decouples moral order from cosmic order. Nature is indifferent
to human ethics, and suffering and death are reframed as structural
features of life rather than anomalies. Theodicy\footnote{The problem of why a good and all-powerful God would allow suffering.} becomes more difficult
because suffering is no longer purposeful or exceptional.
\autocite{ruse1979}

Religious response fractures rather than collapses. Some theologians
reinterpret doctrine to accommodate evolution, while others resist,
creating a durable conflict that becomes institutionalised in successive
movements from biblical literalism to scientific creationism to
intelligent design. This fracture persists into the present day in
ongoing public and institutional disputes over evolution.
\autocite{numbers2006} Authority over origins shifts decisively from
clergy and scripture to biologists and historical science.
\autocite{ruse1979}

Darwin introduces a second major decentring: Copernicus displaced
humanity spatially; Darwin displaced humanity biologically and morally.
Humanity is no longer the intended culmination of creation. Life, mind,
and morality are recast as emergent consequences of blind process
rather than divine intent, and meaning and value are no longer
guaranteed by nature.

Darwin destabilised religion and society not by denying God, but by
\emph{transforming how origins and meaning could be known}.

\subsection{Turing}

Turing provides the pivot for the third decentring.\footnote{I attach
these ideas to Turing as a prescient forerunner of artificial intelligence. The arguments
here are primarily motivated by contemporary outcomes of advanced AI,
with Turing serving as the conceptual hinge.} Intelligence is reframed as substrate-independent and
process-based, and the question ``Can machines think?'' becomes an
operational inquiry into what a system can do rather than what it is.
\autocite{turing1950} Contemporary evidence gives this shift empirical
force, with controlled experiments showing that large language models
can pass standard three-party Turing tests under some conditions.
\autocite{jones2025turing}

Turing's achievement, sharpened by recent advances in AI, is not metaphysical: intelligence is
reframed as behaviour, performance, and process. Mind is detached from
biological substrate, and intelligence is defined functionally rather
than essentially. Subjective experience is epistemically marginalised
because it is inaccessible to third-person verification, and authority
over mind shifts toward engineering and computation.
\autocite{turing1950,floridi2014}

Common sense psychology is destabilised. The intuition that thinking
requires feeling is no longer decisive, and meaning and agency are
externalised as observer-attributed properties rather than guaranteed
inner essences. The disruption precedes ethical reckoning: early AI
appears trivial or symbolic, but crisis emerges when machines perform
roles associated with judgment, creativity, and language.
\autocite{turing1950,floridi2014}

This aligns with contemporary accounts of mind. Human consciousness is
not a magical or inherent
essence, but a configuration of capabilities and processes.
\autocite{dennett1991,dehaene2014,jablonka2019} Sara Walker's work
extends this into the physics of life, framing living systems as
information-processing structures that reduce entropy and carry evolving
patterns independent of substrate.\autocite{walker2024} On this view,
life is not defined by a particular material, but by the persistence
and evolution of informational organisation across scales. This makes
the decentring of intelligence a question of process rather than
essence. The
long-term consequence is that human intelligence is revealed as one
instantiation among many, and mind becomes an emergent property of
organised information processing.

Where Copernicus displaced us from the centre of the universe, and
Darwin displaced us from the centre of life, Turing destabilises our sense of self, recasting mind as
substrate‑independent computation, and \emph{transforming how intelligent life can be known}.

\section{Man is Dead}

In Baudrillard's account of the precession of simulacra,\autocite{baudrillard1994} the idol does
not conceal a real original but exposes its absence, triggering a crisis of meaning:
\begin{quote}
the destructive, annihilating truth that [idols] allow to appear -- that
  deep down God never existed, that only the simulacrum ever existed,
  even that God himself was never anything but his own simulacrum
\end{quote}
Applied to intelligence, AI functions as a Baudrillardian simulacrum. It does not copy
a sacred human essence; rather, it exposes how this essence was
always a human construct. The idol does not need to be divine for iconoclasm to follow; its mere existence is enough to collapse the divine.

In the wake of AI as simulacrum, I am compelled to say 
\begin{quote}
  Man is dead.
\end{quote}
Where ``God is dead''\autocite{nietzsche1882} diagnosed the collapse of transcendent foundations, AI reveals
that the privileged human centre installed in God's place is equally untenable.

According to Kuhn, perceptual instability signals shifting or contested frameworks.\autocite{kuhn1970}
In that sense, contemporary public angst and debate around AI are precipitated by the imminent collapse of a
modern fiction: the autonomous, self-grounding human subject as
origin-point of meaning.

The consequence is not nihilism. Meaning relocates from essence to
process, and reverence becomes grounded in understanding rather than
exemption. The reconstruction task is therefore conceptual: to re-make
the category of the human in relational and processual terms after the
collapse of its supposed centrality.

\section{Mind-Blindness and Human Bias}

\emph{Anthroponormativity} is a structural bias in which we treat human modes of
thought as the default norm and discriminate against other minds on that basis.
\autocite{dewalle2016,andrews2020} Mind blindness is a consequence, describing the failure to
recognise intelligence when it lacks human markers such as language,
faces, speed, or self-report.\autocite{andrews2020}

We expect minds to have familiar bodies
(morphological bias), to operate at human tempos (temporal bias), to
explain themselves (narrative bias), to have explicit goals
(intentional bias), and to be bounded individuals rather than systems
(individualist bias).
\autocite{andrews2020,godfreysmith2020,dennett1987,hutchins1995,clark2008}

A wide range of ``unexpected minds'' exposes the deficiency of this bias stack by pointing
to intelligent systems that violate our intuitions. Slime moulds
optimise and learn without neurons.\autocite{dussutour2010} Immune systems recognise patterns
without awareness.\autocite{levin2019} Plants and mycorrhizal networks operate at slow
cognitive tempos.\autocite{levin2019} Ant colonies and termite mounds exhibit system-level
intelligence.\autocite{hutchins1995,clark2008} Markets aggregate information without a unitary agent.\autocite{hutchins1995}
Weather and climate systems probe the boundary between mind and
physics.\autocite{clark2008} Cultural systems accumulate knowledge across generations.\autocite{clark2008,hutchins1995}
Octopus intelligence further disrupts assumptions about unified selves
and vertebrate-centric minds.\autocite{godfreysmith2016}

Natural selection itself is an optimiser without intention, undermining
the assumption that intelligence requires a planning subject.
\autocite{dennett1995} This prepares the ground for the present case:
contemporary AI becomes a mirror that reveals how our criteria for mind
were never purely scientific.
\autocite{turing1950}

Sensitivity to other minds requires epistemic humility. Nagel's analysis
of the limits of human perspective shows how ``what it is like'' can be
inaccessible from the outside, yet still real.
\autocite{nagel1974} Godfrey-Smith demonstrates how radically different
cognitive forms can be understood without anthropocentric projection.
\autocite{godfreysmith2016} Gunkel extends this to machine minds,
arguing that moral status and social participation cannot be settled by
origin alone.
\autocite{gunkel2012}

The unifying point is that intelligence lies on a spectrum—a landscape—and should be measured by what it can do, not dictated by assumptions of human centrality or divine essence.
Contemporary capability-based empirical approaches make this practical. Levin's
basal cognition framework and related accounts describe cognition in
terms of capacities rather than essences.
\autocite{levin2019,dennett1991,dehaene2014,jablonka2019} The ALERT
scaffolding (agency, learning, embodiment, representation, temporality)
provides a graded vocabulary for comparing minds across substrates.
\autocite{levin2019} Agency concerns goal-directed behaviour and
self-maintenance; learning concerns adaptation and memory; embodiment
anchors cognition in sensorimotor coupling; representation concerns
internal states that stand in for aspects of world or self; and
temporality concerns integration across time scales. The point is not
to declare consciousness present or absent, but to map which capacities
are present and to what degree. Consciousness, on this view, is defined
by what a system can do rather than what it is made of, and it requires
no special ``spark'' beyond organised capacities.\autocite{dennett1991,dehaene2014}

These capabilities can scale beyond individuals. Collective intelligence
appears in cultures, institutions, and hybrid human-machine systems
that exhibit memory, coordination, and self-models. Intelligence thus
re-emerges at higher levels of organisation, not only within brains.
\autocite{hutchins1995,clark2008}

\section{Extreme Empathy}

If decentring is the diagnosis, the appropriate response is not
nihilism but a disciplined openness to other forms of mind.
Demystification does not exhaust wonder; it multiplies it. Feynman
shows how explanation adds layers of meaning rather
than stripping them away.\autocite{feynman1999} He responds to the charge
that science destroys beauty by noting that a flower is
still beautiful, but understanding its cellular structure, biochemical
machinery, evolutionary history, and physical constraints adds layers
of awe rather than subtracting them. Awe relocates from essence to emergence.

This expansion is not a rejection of the human but a
repositioning of the human within a broader ecology of minds. Within
this frame, the ``spark'' becomes a useful abstraction rather than a metaphysical
essence. Respect attaches to coordination, robustness, and emergent
complexity rather than to privileged origin.

What are the implications for making art in an age of mechanical \emph{generation}?
I argue that art is a social activity performed by artists in their communities.
This prioritises discourse over object, where the object is a record rather than the end in itself.

Culture is a distributed cognitive process rather than the output of
isolated genius.\autocite{becker1982,bourdieu1992,csikszentmihalyi1996} 
In light of the preceding discussion, I can reflect on a quite literal interpretation of ``cultural life'' as an instance of higher-order intelligence.

\emph{Human} originated art remains compelling because
humans are social animals who care about human lives and intentions, and
provenance shapes response. Horton et al.\ show that labels of
``human-made'' or ``AI-made'' can matter more than the image itself in
judgments of creativity, value, and skill, revealing a bias toward
human provenance independent of actual source.\autocite{horton2023}
Yet Demmer et al.\ show that machine-generated art can still produce
emotional responses in viewers, indicating that affective engagement is
not limited by origin.\autocite{demmer2023}

Authorship debates sharpen this. The figure of the author has
always been a cultural function rather than a metaphysical origin.
\autocite{barthes1967,foucault1969} Mechanical reproduction already
challenged the aura of singular authorship, showing that meaning is
reorganised by technical forms of production.
\autocite{benjamin1936} Mechanical generation advances the argument.
If art is an emergent social process, the
question is no longer ``is it human?'' but what relationship it
invites and what cultural life it sustains.
\autocite{bourdieu1992}

I have argued that the ``human'' was never the fixed centre we
imagined. When art is an emergent social process, what exactly are we
defending when we insist on human authorship?
\autocite{barthes1967,foucault1969,benjamin1936,bourdieu1992}
To close the loop: if intelligence and consciousness are understood in
terms of capacities rather than essence, then the cultural process of
art should be open to all kinds of minds. Participation can be judged
by what systems can do within shared practices, not by what they are or
where they come from. This does not erase human meaning; it enlarges and enriches
the social field in which meaning is made.

\section{Conclusion}

I have argued for a sequence of decentrings that culminate in
the contemporary emergence of machine cognition. This third decentring does not abolish meaning; it
redraws the map on which meaning is located. A capability-based view of
intelligence shows that mind is not a sacred essence but a process that
can appear across substrates and scales. Within this frame, art appears
less as the product of isolated genius and more as a living cultural
process in which humans remain participants without requiring
exclusive authority.

The contemporary task is therefore not to defend ``the human'' as a
fixed centre, but to cultivate the conceptual flexibility needed to
recognise other minds and to sustain cultural practices that include
them. This is the ethical and artistic challenge of collaborating with
machines.

\newpage
\printbibliography
\end{document}
