\documentclass[12pt]{article}
\usepackage{graphicx} % Required for inserting images
\usepackage{setspace}
\usepackage[utf8]{inputenc}
\usepackage{epigraph}
\usepackage{titlesec}

\titleformat{\section}  % redefine \section
  [block]               % shape
  {\filcenter\normalfont\scshape} % format: centered, normal weight, small caps
  {}                    % label (empty → unnumbered)
  {0pt}                 % separation between label and title
  {}                    % before-code

\titleformat{\subsection}
  [block]
  {\raggedright\normalfont\scshape\normalsize}
  {}
  {0pt}
  {}

\titleformat{\subsubsection}
  [block]
  {\raggedright\normalfont\itshape}
  {}
  {0pt}
  {}

\makeatletter
\renewcommand{\maketitle}{%
  \begin{center}
    {\LARGE\scshape\@title \par} % big small caps title
    \vskip 1em
    {\large\scshape\@author \par} % smaller small caps author
    \vskip 1em
    {\normalsize\@date \par} % date in normal size
  \end{center}
}
\makeatother

\setlength{\epigraphwidth}{\textwidth}
\setlength{\epigraphrule}{0pt}
\renewcommand{\textflush}{flushleft}
\renewcommand{\epigraphsize}{\itshape\normalsize}
\renewcommand{\epigraphflush}{flushleft}
\newcommand{\epigraphsource}[1]{--- {\small{\textit{#1}}}}

\usepackage[
  backend=biber,
  notes,
  dashed=false,
  doi=false,
  url=true,
  isbn=false,
  giveninits=true
]{biblatex-chicago}
\addbibresource{main.bib}

\interfootnotelinepenalty=10000

\title{Anthroponormativity}
\author{Alex Grant}
\date{\today}

\begin{document}
\onehalfspacing
\maketitle

\begin{abstract}

\end{abstract}

\section{Three Great De-centerings}

\subsection{Copernicus}

% Historical positioning and pre-copernicus worldview
% What Copernicus revealed (and how)
% Consequences: (1) religious/social/political (2) Scientific 2(a) epicycles and the almost-discovery of Fourier series by greek scientists - simplified by right frame of reference (2b) Einstein and consequences of a simple assumption

\subsection{Darwin}

% Historical positioning and pre-darwin worldview
% What Darwin revealed (and how)
% Consequences (1) religious/social/political/ (2) scientific

\subsection{Turing}

% Proposing Turing as the pivot point for the decentering of human intelligence via machine cognition - which was only theoretical in his day, but becoming more and more likely today
% What can we learn from previous de-centerings and what do we expect, what do we see happening

% This third decentering reveals that **human consciousness is not a magical or inherent essence**, but an **emergent process**
% Sara Walker and life as information processing or entropy reduction by complex systems. The substrate does not matter, rather it is the evolving information patterns they carry.

\section{Man is Dead}

% Baudrillard - The Precession of Simulacra
% Forbidding of idols - motivated by the idea that divinity cannot be represented, but really predicting the omnipotence of the simulacra
\begin{quote}
  the desctuctive, annihilating truth that they allow to appear -- that deep down God never existed, that only the simulacrum ever existed, even that God himself was never asnything but his own simulacrum
\end{quote}% p. 4

% Application of this idea to intelligence: once we have a simulacrum, it bevomes obvious that intelligence was never special in the first place
% Leads to extension of Nietzsche: "man is dead" by which I mean that our idea of the expectional human was always fictional. 
% This is not a biological claim, but a conceptual one: the modern idea of “Man” as a privileged, self-grounding subject has collapsed.

%- “God is dead” is **not a statement of atheism**, but a diagnosis:
%  - The collapse of transcendent foundations of meaning.
%  - Humanity temporarily replaces God as the source of value and meaning.
%- This substitution is unstable and sets the stage for the next decentering.
% Apply this to the notion that "man is dead"

% The consequence: we need to re-construct "man"

% Nietzsche removes transcendence.
% Baudrillard removes the final illusion of an authentic human core.

\section{Mind-Blindness and Human Bias}
% alert.md
% Capability approach and ALERT Michael Levin and the utilitarian approach to intelligence
% bias.md
% Unexpected intelligences
% Tiers of human bias

\section{Extreme Empathy}
% wonder.md
% "spark" as useful abstraction
% Our response: extreme empathy for all minds
% Art is a social activity perfromed by artists

\newpage
\printbibliography
\end{document}
